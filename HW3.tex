%!TEX program = xelatex
\documentclass{article}
\usepackage[UTF8]{ctex}

\usepackage{amsmath}
\usepackage{amsfonts}
\usepackage{amssymb}
\usepackage{graphicx}
\usepackage{xcolor}
\usepackage[margin=1in]{geometry}
\usepackage{multirow}
\title{Homework3}
\author{21307099 李英骏}
\date{2023.11.1}
\begin{document}
\maketitle

\section{Problem 1}
\subsection{Correlating Branch Predictor}

\begin{table}[h!]
\centering
\resizebox{0.5\textwidth}{!}{%
\begin{tabular}{|c|c|c|c|c|}
\hline
\textbf{Branch PC} & \textbf{Entry} & \textbf{Prediction} & \textbf{Outcome} & \textbf{Wrong} \\ \hline
    2 & 4 & T & T & NO \\ 
    3 & 6 & NT & NT & NO \\
    1 & 2 & NT & NT & NO \\
    3 & 7 & NT & NT & MO \\
    1 & 3 & T & NT & YES \\
    2 & 4 & T & T & NO \\
    1 & 3 & T & NT & YES \\
    2 & 4 & T & T & NO \\
    3 & 7 & NT & T & YES \\ \hline
\end{tabular}
}
\caption{P1-1}
\end{table}

$\text{error rate}=\frac{3}{9} \times 100\% \approx 33.3\%$
\subsection{Local Predictor}

\begin{table}[h!]
\centering
\resizebox{0.5\textwidth}{!}{%
\begin{tabular}{|c|c|c|c|c|}
\hline
\textbf{Branch PC} & \textbf{Entry} & \textbf{Prediction} & \textbf{Outcome} & \textbf{Wrong} \\ \hline
    0 & 0 & T & T & NO \\
    1 & 4 & T & NT & YES \\
    1 & 1 & NT & NT & MO \\
    1 & 3 & T & NT & YES \\
    1 & 3 & T & NT & YES \\
    0 & 0 & T & T & NO \\
    1 & 3 & NT & NT & NO \\
    0 & 0 & T & T & NO \\
    1 & 5 & T & T & NO \\
\end{tabular}
}
\caption{P1-2}
\end{table}

$\text{error rate}=\frac{3}{9} \times 100\% \approx 33.3\%$


\section{Problem 2}

\begin{table}[h!]
\centering
\begin{tabular}{|c|c|c|c|c|c|c|c|}
\hline
Op & dest & j & k & Issue & Read Oper & Exec Comp & Write Result \\
\hline
    LD & F6 & 34 & R2 & 1 & 2 & 3 & 4 \\
    LD & F2 & 45 & R3 & 5 & 6 & 7 & 8 \\
    MULD & F0 & F5 & F2 & 6 & 9 & 19 & 20 \\
    MULD & F7 & F2 & F6 & 6 & 9 & 19 & 20 \\
    ADDD & F6 & F8 & F7 & 6 & 21 & 23 & 24 \\
\hline
\end{tabular}
\caption{P2(a)}
\end{table}

\begin{table}[h!]
\centering
\begin{tabular}{|c|c|c|c|c|c|c|c|c|c|c|}
\hline
Time & Name & Busy & Op & \(F_i\) & \(F_j\) & \(F_k\) & \(Q_j\) & \(Q_k\) & \(R_j\) & \(R_k\) \\
\hline
    9 & Integer & No & & & & & & & &\\
    9 & Mult1 & Yes & MULD & F0 & F5 & F2 & & Integer & Yes & Yes \\
    9 & Mult2 & Yes & MULD & F7 & F2 & F6 & Integer & Integer & Yes & Yes \\
    9 & Add & Yes & ADDD & F6 & F8 & F7 & & Mult2 & & No \\
\hline
\end{tabular}
\caption{P2(b)}
\end{table}

\section{Problem 3}

\begin{enumerate}
  \item \textbf{Cost of BTB miss:} \\
  The cost for BTB miss events, which occur for the 15\% of branch instructions not hit in the BTB:
  \begin{equation*}
    \text{BTB miss cost} = \text{Branch frequency} \times \text{Miss rate} \times \text{Miss penalty}
  \end{equation*}
  \begin{equation*}
    = 20\% \times 15\% \times 3 \text{ cycles} = 0.09 \text{ cycles/instruction}
  \end{equation*}

  \item \textbf{Cost for hits with wrong prediction:} \\
  The cost for cases where BTB hits but the prediction is wrong, which happens for 10\% of the 85\% of branch instructions where the BTB hits:
  \begin{equation*}
    \text{Hit but wrong prediction cost} = \text{Branch frequency} \times \text{Hit rate} \times \text{Wrong prediction rate} \times \text{Misprediction penalty}
  \end{equation*}
  \begin{equation*}
    = 20\% \times 85\% \times 10\% \times 4 \text{ cycles} = 0.068 \text{ cycles/instruction}
  \end{equation*}

  \item \textbf{Total CPI including branch prediction:} \\
  The total CPI including branch prediction consists of the base CPI plus the additional CPI due to branch prediction overhead:
  \begin{equation*}
    \text{Total CPI} = \text{Base CPI} + \text{BTB miss cost} + \text{Hit but wrong prediction cost}
  \end{equation*}
  \begin{equation*}
    = 1 + 0.09 + 0.068 = 1.158 \text{ cycles/instruction}
  \end{equation*}

  \item \textbf{CPI for processor without BTB:} \\
  For a processor without a BTB, there is a fixed two-cycle penalty for each branch instruction:
  \begin{equation*}
    \text{CPI without BTB} = \text{Base CPI} + \text{Branch frequency} \times \text{Fixed branch penalty}
  \end{equation*}
  \begin{equation*}
    = 1 + 20\% \times 2 = 1.4 \text{ cycles/instruction}
  \end{equation*}

  \item \textbf{Speedup:} \\
  The speedup is the ratio of the CPI without BTB to the CPI with BTB:
  \begin{equation*}
    \text{Speedup} = \frac{\text{CPI without BTB}}{\text{CPI with BTB}}
  \end{equation*}
  \begin{equation*}
    \approx \frac{1.4}{1.158} \approx 1.209
  \end{equation*}
\end{enumerate}

\section{Problem 4}

\subsection*{(a)}
When a hit occurs, it directly leads to executing the next instruction without any delay, effectively decreasing the cycle count by one per branch instruction. Therefore, the penalty adjustment for a buffer hit with an unconditional branch is:

\begin{equation*}
    \text{Penalty adjustment for buffer hit} = -1 \text{ clock cycle per unconditional branch instruction}
\end{equation*}

\subsection*{(b)}
Assuming an 80\% hit rate for unconditional branches and a frequency of 10\% occurrence in instruction set, the calculation for performance improvement from branch folding is as follows:

\begin{itemize}
  \item For buffer hits (80\% of the time), the enhancement saves one clock cycle per hit.
  \item For buffer misses (20\% of the time), the standard penalty is two clock cycles per miss.
\end{itemize}

The expected improvement from branch folding can therefore be calculated by:

\[
\begin{aligned}
    \text{Expected improvement} &= \text{Frequency of unconditional branches} \times \\
    &(\text{Hit rate} \times \text{Improvement per hit} + \text{Miss rate} \times \text{Penalty per miss})\\
    &= 10\% \times (80\% \times (-1 \text{ cycle}) + 20\% \times 2 \text{ cycles})\\
    &= 10\% \times (-0.8 \text{ cycles} + 0.4 \text{ cycles})\\
    &= -0.04 \text{ cycles per instruction}
\end{aligned}
\]

The negative sign indicates a reduction in the total cycle count per instruction, which is a performance gain. For conditional branches with an assumed two-cycle buffer miss penalty and the same calculation method, the cost would be:
\[
\begin{aligned}
    \text{Penalty for conditional branches storing target address} &= 10\% \times (80\% \times 0 + 20\% \times 2 \text{ cycles})\\
    &= 0.04 \text{ cycles per instruction}
\end{aligned}
\]
The overall performance gain due to the BTB enhancement is:
\begin{equation*}
    \text{Overall performance gain} = (0.04 - (-0.04)) \text{ cycles per instruction}= 0.08 \text{ cycles per instruction improvement}
\end{equation*}


\newpage
\section{Problem 5}

\begin{table}[h!]
\centering
\begin{tabular}{|c|l|c|c|c|c|l|}
\hline
Iteration & Instructions & Issue & Executes & Memory access & Write CDB & Comment \\ \hline
    \multirow{2}{*}{1} & LD F2,0(R1) & 1 & & 2 & 3 & 1st instruction \\ \cline{2-7} 
    & MUL.D F4,F2,F0 & 2 & 4 & & 19 & \begin{tabular}[c]{@{}l@{}}Awaiting F2; 3-\\ 4: Reservation Station; 5-\\ 18: Multiply execution\end{tabular} \\ \hline
    % Continue with other rows in a similar fashion
    & L.D F6,0(R2) & 3 & & 4 & 5 & 4: Load buffer \\ \cline{2-7} 
    & ADD.D F6,F4,F6 & 4 & 20 & & 30 & \begin{tabular}[c]{@{}l@{}}Awaiting F4; 5-\\ 20: Reservation Station; 21-\\ 29: Add execution\end{tabular} \\ \cline{2-7} 
    & S.D F6,0(R2) & 5 & & 31 & & \begin{tabular}[c]{@{}l@{}}Awaiting F6; 6-\\ 31: Store buffer\end{tabular} \\ \cline{2-7} 
    & DADDIU R1,R1,\#8 & 6 & 7 & & 8 & \\ \cline{2-7} 
    & DADDIU R2,R2,\#8 & 7 & 8 & & 9 & \\ \cline{2-7} 
    & DSLTU R3,R1,R4 & 8 & 9 & & 10 & Awaiting R3 \\ \cline{2-7} 
    & BNEZ R3,foo & 9 & 11 & & & Awaiting BNEZ; \\ \cline{2-7} 
    \multirow{2}{*}{2} & L.D F2,0(R1) & 10 & & 12 & 13 & \begin{tabular}[c]{@{}l@{}}Awaiting BNEZ; \\ 11-12: Load buffer\end{tabular} \\ \cline{2-7} 
    & MUL.D F4,F2,F0 & 11 & 19 & & 34 & \begin{tabular}[c]{@{}l@{}}Awaiting F2; 13-\\ 19: Reservation Station; 20-\\ 33: Multiply execution\end{tabular} \\ \cline{2-7} 
    & L.D F6,0(R2) & 12 & & 13 & 14 & 13: Load buffer \\ \cline{2-7} 
    & ADD.D F6,F4,F6 & 13 & 35 & & 45 & \begin{tabular}[c]{@{}l@{}}Awaiting F4; 14-\\ 35: Reservation Station; 36-\\ 44: Add execution\end{tabular} \\ \cline{2-7} 
    & S.D F6,0(R2) & 14 & & 46 & & \begin{tabular}[c]{@{}l@{}}Awaiting F6; 15-\\ 46: Store buffer\end{tabular} \\ \cline{2-7} 
    & DADDIU R1,R1,\#8 & 15 & 16 & & 17 & \\ \cline{2-7} 
    & DADDIU R2,R2,\#8 & 16 & 17 & & 18 & \\ \cline{2-7} 
    & DSLTU R3,R1,R4 & 17 & 18 & & 20 & Awaiting R3 \\ \hline
    & BNEZ R3, foo & 18 & 20 & & & AWaiting R3 \\ \hline
\end{tabular}
\caption{P5}
\end{table}


\end{document}