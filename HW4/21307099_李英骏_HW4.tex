%!TEX program = xelatex
\documentclass{article}
\usepackage[UTF8]{ctex}

\usepackage{amsmath}
\usepackage{amsfonts}
\usepackage{amssymb}
\usepackage{amsthm}
\usepackage{algobox}
\usepackage{graphicx}
\usepackage[margin=1in]{geometry}
\usepackage{listings}
\usepackage{multirow}
\usepackage{xcolor}

\title{Homework3}
\author{21307099 李英骏}
\date{2023.11.27}

\begin{document}
\maketitle

\section*{迟交原因说明}
上周阳了,休息了三四天, 还需要补数据库等课程的作业, 因此迟交一天左右, 非常抱歉。已提前和伍学长说明, 准假单已附在文件中。还请老师和助教理解。

\section*{Problem 1}
\subsection*{(a)}
\begin{enumerate}
    \item 基础吞吐量
          \begin{itemize}
              \item 每个\textit{nondiverged SIMD instruction}每4个周期产生32个结果
              \item 每个\textit{SIMD processor}包含8个通道, 说明每个通道每4个周期可以产生$\frac{32}{8} = 4$个结果
              \item 考虑到80\%的线程活跃,每条指令实际产生的结果数为$32 \times 0.8 = 25.6 \text{Flops}$
              \item 频率为\textit{2 GHz},则每秒执行$\frac{2 \, \text{G cycles}}{4} = 0.5 \, \text{G cycles}$组指令
              \item 考虑到共10个处理器, 基础吞吐量为$0.5 \,\text{G/s} \times 25.6 \,\text{G/s} \times 10 = 128 \,\text{GFLOPS/s}$
          \end{itemize}

    \item 考虑指令类型和发行率
          \begin{itemize}
              \item 70\%的指令为单精度算术,且\textit{average SIMD instruction issue rate}为0.85, 则该内核在该GPU上的吞吐量为:
                    \[ 128 \,\text{GFLOPS/s} \times 70\% \times 0.85 = 76.16 \,\text{GFLOPS/s} \]
          \end{itemize}
\end{enumerate}

\subsection*{(b)}
\begin{enumerate}
    \item Increasing the number of single-precision lanes to 12
          \begin{itemize}
              \item 若使用线性比例计算吞吐量的提升(某些GPU可能优化了指令宽度和通道数量不对齐的情况), $ Speedup = \frac{12}{8} = 1.5$
              \item 但由于32不能被12整除, 即长度为32的向量被分为12, 12, 8三个周期处理, 而原先需要4个周期。
                    则加速比为$ Speedup = \frac{4}{3} \approx 1.3333 $
          \end{itemize}

    \item Increasing the number of SIMD processors to 15 (assume this change doesn't
          affect any other performance metrics and that the code scales to the additional
          processors)
          \begin{itemize}
              \item 假设增加处理器数量不影响其他性能指标, 并且代码可以扩展到额外的处理器, 则\\
                    $Speedup = \frac{15}{10} = 1.5$
          \end{itemize}

    \item Adding a cache that will effectively reduce memory latency by 40instruction
          issue rate to 0.95\%, which will increase
          \begin{itemize}
              \item 缓存减少内存延迟40\%,提高\textit{instruction issue rate}到0.95,
                    则 新的吞吐量为:\\$76.16 \,\text{GFLOPS/s} \times \frac{0.95}{0.85} = 85.12 \,\text{GFLOPS/s}$
              \item $Speedup = \frac{0.95}{0.85} \approx1.1176 $
          \end{itemize}

\end{enumerate}

\section*{Problem 2}
Assume that both the vector processor and GPU are
performance bound by memory bandwidth. 因此认为向量处理器的性能足够高, 忽略向量处理器的执行时间。
\subsection*{向量计算机执行时间}
\begin{itemize}
    \item 标量部分的执行时间为 300 毫秒。
    \item 向量内核的执行时间:由于向量处理器的内存带宽为 30 GB/s,内核处理 400 MB 数据(300 MB输入 + 100 MB输出)。
    \item $ \textit{向量内核执行时间} = \frac{400 \,\text{MB}}{30 \,\text{GB/s}} = \frac{400 \,\text{MB}}{30\times 2^{10} \,\text{MB/s}} \approx13.0208\,\text{ms}$。
    \item $ \textit{总执行时间} = 300 \,\text{ms} + 13.0208 \,\text{ms}= 313.0208 \,\text{ms}$
\end{itemize}

\subsection*{混合计算机执行时间}
\begin{itemize}
    \item 标量部分的执行时间为 300 毫秒,这是在主机处理器上执行的,时间是给定的,无需计算。
    \item 输入向量从主存载入GPU本地内存
          \begin{itemize}
              \item 主机内存 -> GPU本地内存
              \item 300 MB
              \item 使用DMA带宽(10 GB/s):\( \frac{300\,\text{MB}}{10\,\text{GB/s}} = 29.2969\,\text{ms}\)
          \end{itemize}
    \item 向量内核在GPU上执行:
          \begin{itemize}
              \item GPU内存 -> GPU处理器 -> GPU内存
              \item 输入300 MB + 输出100 MB = 400 MB
              \item 使用GPU内存带宽(150 GB/s):\( \frac{400\,\text{MB}}{150\,\text{GB/s}} = 2.6042\,\text{ms}\)
          \end{itemize}
    \item 输出向量从GPU本地内存传回主存:\\
          题中写的是requires all \textit{input vectors} to\dots, 我理解应该是相对的, 即运算完传输的是输出向量。
          \begin{itemize}
              \item GPU本地内存 -> 主机内存
              \item 100 MB
              \item 使用DMA带宽(10 GB/s):\( \frac{100\,\text{MB}}{10\,\text{GB/s}} = 9.7656\,\text{ms}\)
          \end{itemize}
    \item DMA传输延迟:
          \begin{itemize}
              \item 由于有两次传输(输入和输出),总延迟为20毫秒
          \end{itemize}
    \item 总执行时间:
          \[300\,\text{ms} + 29.2969\,\text{ms} +
              2.6042\,\text{ms} + 9.7656\,\text{ms} +
              20\,\text{ms}\approx361.67\,\text{ms}\]
\end{itemize}

\section*{Problem 3}
\subsection*{(a)}
Assume that when processors are in use, the applications run times faster.
故修正Amdahl定律如下:
\[
    S(p) = \frac{1}{\sum_{i=1}^{p} \frac{F(i, p)}{i}}
\]

\subsection*{(b)}
由上述修正Amdahl定律:
\begin{align*}
    S(8) & = \frac{1}{\left(F(1, 8) \cdot 1 + F(2, 8) \cdot \frac{1}{2} + F(4, 8) \cdot \frac{1}{4} + F(6, 8) \cdot \frac{1}{6} + F(8, 8) \cdot \frac{1}{8}\right)}                 \\
         & = \frac{1}{\left(\left(1-\sum_{i=2}^{8}F(i, p)\right) \cdot 1 + 20\% \cdot \frac{1}{2} + 10\% \cdot \frac{1}{4} + 5\% \cdot \frac{1}{6} + 15\% \cdot \frac{1}{8}\right)} \\
         & = \frac{1}{\left(50\% \cdot 1 + 20\% \cdot \frac{1}{2} + 10\% \cdot \frac{1}{4} + 5\% \cdot \frac{1}{6} + 15\% \cdot \frac{1}{8}\right)}                                 \\
         & \approx1.5335
\end{align*}

\subsection*{(c)}
由上述修正Amdahl定律, 对于32处理器的情况:
\begin{align*}
    S(32) & = \frac{1}{\left(F(1, 32) \cdot 1 + F(2, 32) \cdot \frac{1}{2} + F(4, 32) \cdot \frac{1}{4} + F(6, 32) \cdot \frac{1}{6} + F(8, 32) \cdot \frac{1}{8} + F(16, 32) \cdot \frac{1}{16}\right)}        \\
          & = \frac{1}{\left(\left(1-\sum_{i=2}^{32}F(i, p)\right) \cdot 1 + 20\% \cdot \frac{1}{2} + 10\% \cdot \frac{1}{4} + 5\% \cdot \frac{1}{6} + 15\% \cdot \frac{1}{8} + 20\% \cdot \frac{1}{16}\right)} \\
          & = \frac{1}{\left(30\% \cdot 1 + 20\% \cdot \frac{1}{2} + 10\% \cdot \frac{1}{4} + 5\% \cdot \frac{1}{6} + 15\% \cdot \frac{1}{8} + 20\% \cdot \frac{1}{16}\right)}                                  \\
          & \approx2.1525
\end{align*}
对于无穷个处理器的情况:
\begin{align*}
    S(\infty) & = \frac{1}{\left(F(1, \infty) \cdot 1
        + F(2, \infty) \cdot \frac{1}{2}
        + \cdots
    + F(\infty, \infty) \cdot \frac{1}{\infty}\right)}                                                                                                                                                              \\
              & = \frac{1}{\left(F(1, \infty) \cdot 1
        + F(2, \infty) \cdot \frac{1}{2}
        + F(4, \infty) \cdot \frac{1}{4}
        + F(6, \infty) \cdot \frac{1}{6}
        + F(8, \infty) \cdot \frac{1}{8}
        + F(16, \infty) \cdot \frac{1}{16}
    + F(128, \infty) \cdot \frac{1}{128}\right)}                                                                                                                                                                    \\
              & = \frac{1}{\left(20\% \cdot 1 + 20\%  \cdot \frac{1}{2} + 10\%  \cdot \frac{1}{4} + 5\%  \cdot \frac{1}{6} + 15\%  \cdot \frac{1}{8} + 20\%  \cdot \frac{1}{16} + 10\%  \cdot \frac{1}{128}\right)} \\
              & \approx2.7370
\end{align*}

\section*{Problem 4}
\subsection*{(a)}
\begin{lstlisting}[frame=single]
    for (i=0; i < 100; i++) {
        A[i] = B[2*i+4]; /* S1 */
        B[4*i+5] = A[i]; /* S2 */
    }
\end{lstlisting}
$S2$ uses the value \(A[i]\) computed by $S1$ in the same iteration

\subsection*{(b)}
\begin{lstlisting}[frame=single]
    for (i=0; i < 100; i++) {
        A[i] = A[i] * B[i]; /* S1 */
        B[i] = A[i] + c;    /* S2 */
        A[i] = C[i] * c;    /* S3 */
        C[i] = D[i] * A[i]; /* S4 */
    }    
\end{lstlisting}
\begin{enumerate}
    \item True Dependences (Read after Write)
          \begin{itemize}
              \item S1 on A[i] followed by S2 on A[i].
              \item S1 on A[i] followed by S4 on A[i].
              \item S3 on A[i] followed by S4 on A[i].
          \end{itemize}
    \item Output Dependences (Write after Write)
          \begin{itemize}
              \item S1 and S3 on A[i].
          \end{itemize}
    \item Antidependences (Write after Read)
          \begin{itemize}
              \item S1 on B[i] followed by S2 on B[i].
              \item S1 on A[i] followed by S3 on A[i].
              \item S2 on A[i] followed by S3 on A[i].
              \item S3 on C[i] followed by S4 on C[i].
          \end{itemize}
\end{enumerate}
\begin{lstlisting}[frame=single]
    for(i=0;i<100;i++){
        A[i] = A[i] * B[i]; /*S1*/
        B1[i] = A[i] + c; /*S2*/
        A1[i] = C[i] * c; /*S3*/
        C1[i] = D[i] * A1[i]; /*S4*/
    }
\end{lstlisting}

\subsection*{(c)}
\begin{lstlisting}[frame=single]
    for (i=0; i < 100; i++) {
        A[i] = B[i] + C[i]; /* S1 */
        B[i+1] = D[i] + E[i]; /* S2 */
        C[i+1] = D[i] * E[i]; /* S3 */
    }
\end{lstlisting}
\noindent
S1 读取了 S2 在前一轮迭代的写入值. 在iteration i 中S2计算了 B[i+1], 这在 iteration i+1 中被S1读取.
S1 与 S3 中的 C[i] 和 C[i+1]同理, 因此不是并行的
\begin{lstlisting}[frame=single]
    // 使用临时数组来避免写入依赖
    int tempB[100], tempC[100];
    tempB[0] = B[0];
    tempC[0] = C[0];
    
    for (i=0; i < 100; i++) {
        A[i] = B[i] + C[i]; // S1 不变
        if (i < 99) {
            tempB[i+1] = D[i] + E[i]; // S2 写入临时数组
            tempC[i+1] = D[i] * E[i]; // S3 写入临时数组
        }
    }
    
    // 循环结束后,将临时数组的内容复制回原数组
    for (i=1; i < 100; i++) {
        B[i] = tempB[i];
        C[i] = tempC[i];
    }
    
\end{lstlisting}

\section*{Problem 5}
\subsection*{峰值单精度浮算吞吐量}
\noindent
假设所有内存延迟都被隐藏, 则峰值吞吐量仅考虑FPU的运算能力, 为
\[ 16 \times 16\,\text{Flops}\times1.5\,\text{GHz} = 384\,\text{GFlops/s}\]
\subsection*{可持续性}
若想要该吞吐量可持续, 则需要在每次浮点运算时输入2个单精度浮点数, 输出1个单精度浮点数, 由IEEE 754标准, 一个单精度浮点数为32位即4 Bytes。所需的内存带宽为
\[384\,\text{GFlops/s} \times 3\,\text{floats/Flops} \times 4\,\text{Bytes/float} = 4608 \,\text{GB/S} \gg 100\,\text{GB/S}\]
因此不可持续。

\end{document}